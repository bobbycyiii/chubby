\documentclass{article}
\usepackage{amssymb,amsmath,amsthm,graphicx}
\theoremstyle{plain}
\newtheorem{theorem}{Theorem}[section]
\newtheorem{corollary}[theorem]{Corollary}
\newtheorem{lemma}[theorem]{Lemma}
\newtheorem{conjecture}[theorem]{Conjecture}
\newtheorem{proposition}[theorem]{Proposition}
\newtheorem{construction}[theorem]{Construction}

\theoremstyle{plain}
\newtheorem*{claim}{Claim}
\newtheorem*{program}{Program}

\theoremstyle{definition}
\newtheorem{definition}[theorem]{Definition}
\newtheorem{definitions}[theorem]{Definitions}
\newtheorem{notation}[theorem]{Notation}
\newtheorem{example}[theorem]{Example}
\newtheorem{examples}[theorem]{Examples}
\newtheorem{remark}[theorem]{Remark}
\numberwithin{equation}{section}
\newcommand{\MOM}{\textsc{Mom}}
\begin{document}
\title{Combinatorial \MOM\ technology with boundary}
\author{RCH3}
\date{\today}
\maketitle
In the original \MOM\ papers, there is a big
fuss about first trying to show that there is
in fact an embedded handle structure of a given
type, and then to show that one can assume that
this structure is ``full.'' Fullness serves two
purposes: on one hand, it rules out obvious
non-hyperbolicity conditions on the embedded
handle structure; on the other hand, it is
straightforward to enumerate full structures.
However, it was mighty hard to show that we can
assume our structures to be full in the original
sense. I would like to propose a more generous
definition of fullness which lends itself to an
easier analysis, which I present here.

The flexibility comes from allowing ourselves
to base our handle structures on bounded surfaces.
This idea was already used in the original \MOM\ papers
but was exploited more than accepted. One used in those
papers such a handle structure only as a stepping stone
towards a handle structure on a closed surface. In this
whitepaper we accept such structures in themselves.

\begin{definition}\label{defn:init}
If $S$ is a tame surface, then a \emph{\MOM\ handle-structure based on $S$}
is the result of attaching 1-handles and 2-handles to the $S\times \{1\}$
boundary component of $S\times I$, in a sufficiently nice way. We'll want
to be more precise later about how nice we can and should make these structures:
for now, here are some properties we can get and which we like:
\begin{itemize}
\item the 1-handle attaching discs and 2-handle attaching curves are disjoint
      from $(\partial S) \times \{1\}$;
\item the 2-handles are transverse to the 1-handles;
\item all the 2-handles have minimal geometric intersection number with the 1-handles;
\item the intersection of any two 2-handles intersects $S \times \{1\}$ only
      along 1-handle disc boundaries;
\item the boundary components not containing $S\times\{0\}$ are tori.
\end{itemize}

To such a handle structure we naturally associate a graph embedding
$\Gamma \hookrightarrow S (\times \{1\})$, viz. the embedding of the graph formed
by the cores of the 1- and 2- handles. The structure is \emph{full}
when $S \downharpoonright \Gamma$ is a disjoint union of discs and
$\partial$-parallel annuli (or punctured discs, depending on how
you've decided to represent your surface $S$ of finite type).
(End of Definition.)
\end{definition}
\begin{proposition}
It is straightforward to represent, enumerate, and triangulate
full \MOM\ handle structures (even on bounded bases).
\end{proposition}
\proof[Sketch]
Representing, enumerating, and triangulating full structures
on closed bases was done already in the original \MOM\ papers,
and possibly before, e.g. maybe by Matveev and company. What
is new is that we can allow bounded bases.

To represent a full \MOM\ structure on a closed base
requires only a map on the base (a full graph embedding on
the base), together with information about how the vertices'
corresponding 1-handle attaching discs get attached, i.e.
how much the 2-handles twist around the 1-handle. To enumerate
them is a simple matter of enumerating orbit representatives
of deranged involutions on dipyramid faces under the dipyramid
symmetry group. To triangulate them, one just uses the associated
dipyramid gluing.

To represent a full \MOM\ structure on a bounded base, 
consider that we can cap off the base's boundary components
to get a full structure on a closed base. We just use the
corresponding map and twist info, together with information about
which faces of the map come from capping boundary components.
To enumerate these, we just enumerate the corresponding
full closed structures, then enumerate choices of faces to puncture.
To triangulate such a structure, look at the picture boundaryFull.svg
on the Dropbox. Non-equatorial edge-classes of the dipyramidal cellulation
of the closed structure correspond to faces of the map. Puncturing
the faces corresponds to removing such an edge-class. We can accomplish
this by removing a lozenge around the edge-class; and we can do this by
undoing the dipyramidal gluing, and attaching some ``plectrums'' to
triangles adjacent to this edge class, then gluing up the outer triangles
as before, but leaving the new triangles forming a bigon unglued.
(End of Sketch.)

For \MOM\ theory to apply to volume questions, we must begin
with a theorem of the form ``If $N$ is a Foo manifold of geometric
complexity at most Bar, then $N$ admits a nicely embedded handle structure of
combinatorial complexity at most Quz.'' Let us
call such a theorem a \MOM\ theorem. The ``nice embeddings''
are more precisely defined as follows.

\begin{definition}
A \emph{based manifold} is $(X,K)$ where $X$ is a manifold
and $K$ is a boundary-free subsurface of $\partial X$.

Given a handle structure $H$ based on a connected
surface $S$, there is a unique boundary-free subsurface
of $\partial |H|$ containing $S$, where $|H|$ is the
underlying space of $H$. So to any handle structure $H$
we may canonically associate the based manifold $B(H) = (|H|, K)$.

A \emph{based embedding} between two based manifolds 
$(X,K) \hookrightarrow (X',K')$ is simply
an embedding identifying $K$ oriented-homeomorphically with $K'$.

An embedding $f: M \hookrightarrow N$ is \emph{nonelementary}
when the induced map $\pi_1(f)$ has nonabelian image.
(End of Definition.)
\end{definition}

The following is the primary \MOM\ theorem from the JAMS paper:

\begin{theorem}[Thms. 5.9 and 6.1, JAMS paper]
If $N$ a compact orientable 3-manifold with $\partial N = T^2$
admits a finite-volume complete hyperbolic metric on its interior,
and if $vol(N) < 2.848$, then there exists a \MOM\ handle structure
$H$ based on $T^2$ and a nonelementary based embedding 
$B(H) \hookrightarrow (N, \partial N)$, such that $H$ has at most
three 1-handles, and has 2-handles all of valence 3.
\end{theorem}

The proof of this theorem takes up the bulk of the JAMS paper,
using numerical computing and a case analysis thirteen strong.
The following work is not about simplifying this theorem's proof.

However, a sixth of the work of the JAMS paper and the vast 
majority of the Commentarii paper are dedicated to showing that
we may assume these structures are torus-friendly, full,
and hyperbolic. I believe the work presented here gives a
simpler approach to this part of the \MOM\ literature.

Another \MOM\ theorem which Craig, Rob, and I would like
to prove is the following
\begin{conjecture}[from MOM-1.5 Project]
If $N$ a compact orientable 3-manifold
admits a finite-volume hyperbolic metric with totally geodesic
boundary, and $vol(N) < 7.4$, then there exists a \MOM\ handle
structure $H$ based on $\Sigma_2$ and a based embedding
$B(H) \hookrightarrow (N,\partial N)$, such that $H$
has one 1-handle and two valence-3 2-handles.
\end{conjecture}

This promises to be much simpler to prove.

Finally, in the low cusp area project we are hoping to prove
\begin{conjecture}[from Low Cusp Area project]
If $N$ a compact orientable 3-manifold admitting a
complete finite-volume hyperbolic metric on its interior,
and if $N$ has a cusp $T$ with maximal cusp area at most
$5.24$, then there exists a \MOM\ handle structure $H$
based on $T^2$ and a nonelementary based embedding $B(H) \hookrightarrow (N,T)$
such that H has one 1-handle and one 2-handle of valence at most 7.
\end{conjecture}

The rest of this whitepaper is basically an argument
that after proving such theorems, we can get a computer
to do the rest of the work for us.

\section{Fullness}
Assuming we already know how to represent, enumerate,
and triangulate full handle structure of given type,
we would like to promote the handle structure embeddings
coming from our \MOM\ theorems into embeddings of
full such structures. Here is how that works. I'll do
the original case first, where the base is a torus,
and the based embedding is further assumed to be nonelementary.

\begin{proposition}
Suppose $N$ is a hyperbolic (generalized) link complement,
and $T$ is a component of $\partial N$. Suppose that $H$
is a \MOM\ handle structure based on $T^2$, and that
$B(H) \hookrightarrow (N, T)$ is a nonelementary based embedding;
and suppose $\Gamma \hookrightarrow \Xi$ is the fatgraph
coming from the embedding of $H$'s graph into $T^2$.

Then there is a full \MOM\ handle structure $H'$ based on
either $T^2$ or $A \sqcup A$ (the disjoint union of two annuli)
with the same fatgraph $\Gamma \hookrightarrow \Xi$, and
there is a nonelementary based embedding $B(H') \hookrightarrow (N, T)$.
\end{proposition}

\end{document}
